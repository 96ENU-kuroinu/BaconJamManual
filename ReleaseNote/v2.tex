\subsection*{バージョンコード3(1.1.0)}
\addcontentsline{toc}{subsection}{バージョンコード3(1.1.0)}
\begin{itemize}
    \item[リリース日] 2024年4月7日
\end{itemize}

\new
\begin{itemize}
    \item \nowplaying の公開範囲を以下から選択できる機能を追加しました。
        \begin{enumerate}
            \item パブリック(全てのユーザーに公開)
            \item ホーム(ホームタイムラインのみに公開)
            \item フォロワー(自分のフォロワーのみに公開)
            \item チャンネル\footnote{チャンネルIDの設定が必要です}
        \end{enumerate}
    \item Spotifyとの連携機能(プレビュー版)を追加しました。
    \item 設定画面に\ttbox{設定をリセットする}ボタンを追加しました。
\end{itemize}

\change
\begin{itemize}
    \item \nowplaying の際に以下の順にアーティスト名を取得し、最初に見つかったものを表示するように変更しました。
        \begin{enumerate}
            \item アルバムアーティスト名
            \item トラックアーティスト名
            \item 著者名
            \item 作者名
        \end{enumerate}
    \item \mi のアクセストークンが空欄の場合、\ttbox{テスト接続}ボタンを押下できないように変更しました。
    \item 設定画面の各種項目を折りたためるようにUIを変更しました。
\end{itemize}

\fix
\begin{itemize}
    \item ランキング画面の不要な\ttbox{TEST}ボタンを削除しました。
    \item 初回インストール時に投稿パターンの初期値が設定されておらず、\nowplaying できない不具合を修正しました。
\end{itemize}